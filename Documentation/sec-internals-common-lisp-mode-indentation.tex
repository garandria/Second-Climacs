\section{Computing indentation}
\label{sec-internals-common-lisp-mode-indentation}

\subsection{Introduction}

\subsubsection{Definition and unit of indentation}

Indentation refers to the whitespace preceding the first
non-whitespace item of a line of text in a buffer.  For reasons of
backward compatibility, we handle indentation in terms of an integral
number of space characters in a fixed-width font, so that text
read-from and written-to an external file will be compatible with
existing development tools.  At the end of this chapter, we describe
how a buffer containing indented \commonlisp{} code can be displayed
using a proportional font, even though the buffer contents has been
computed for a fixed-with font.

The basic unit of indentation is considered to be two space
characters.  Occasionally, we suggest the use of a single space
characters for certain buffer elements.

\subsubsection{Indentation block}

Indentation is complicated by the presence of comments.  To simplify
the computation, we define an \emph{indentation block} as either:

\begin{enumerate}
\item A maximal sequence of comment wads, each starting with at least
  two semicolons, followed by a single expression wad, followed by a
  maximal sequence of comment wads, each starting with a single
  semicolon.  In this case we say that we have a \emph{complete
    indentation block}.
\item The sequence of all child comment wads at the end of an
  expression wad, if there is a comment wad starting with at least two
  semicolons not followed by an expression wad, as a child of an
  expression wad.  In this case, we say that we have a
  \emph{degenerate indentation block}.
\end{enumerate}

\subsection{Overall indentation vs component indentation}

We consider two different indentation types for a wad in a buffer:

\begin{enumerate}
\item The \emph{overall indentation} of the wad, which is the
  indentation of the first character of the wad relative to the
  overall indentation of its parent wad.
\item The \emph{component indentation} of a compound wad, which
  is the overall indentation of the child wads, of the wad.
\end{enumerate}

Component indentation of a compound wad is determined by the kind of
expression that the compound wad represents.  A compound wad that
represents a special form has rule for computing component indentation
that depends on the special operator for that form, and similarly, for
a component wad that represents a standard macro form.

\subsubsection{Current indentation vs desired indentation}

The \emph{current indentation} of a wad is the current relative
horizontal position of its overall indentation compared to the current
relative horizontal position of its parent.  The current indentation
of a top-level wad is its horizontal position relative to 0.

The \emph{desired indentation} of a wad is the desired relative
horizontal position of its overall indentation compared to the current
relative horizontal position of its parent.  The desired indentation
of a wad depends on the nature of its parent.  Notice that the desired
indentation is independent of the desired indentation of its parent,
and depends only on the current indentation of its parent.

\subsubsection{Basic rules for desired indentation}

The expression wad and the preceding comment wads of a complete
indentation block have the same desired indentation.

If the first wad of a complete indentation block (which can be a
comment wad with at least two semicolons, or an expression wad) is
preceded on the same line by some other wad, then the desired
indentation of the preceding comment wads and the expression wad is
the current indentation of the first wad of the block.

Example:
\begin{verbatim}
(let  ;; The user wanted the bindings to start
      ;; two spaces after LET, for some reason.
      ((x 10))
  ...)
\end{verbatim}

If the first wad of a complete indentation block is not preceded by
any other wad on the same line, then the desired indentation of the
preceding comment wads and the expression wad is determined by the
indentation rules of the parent wad and the nature of the expression
wad in the block.

Example:
\begin{verbatim}
(let  
    ;; The rule for LET is probably 4 spaces for the bindings.
    ((x 10))
   ;; And maybe 3 for declarations?
   (declare (type integer x))
  ;; Certainly 2 for the body.
  (+ x 20))
\end{verbatim}

If the first comment wad following the expression wad in a complete
indentation block is preceded by the expression wad on the same line,
then the desired indentation for the comment wad is its current
indentation.  Subsequent comment wads have the same desired
indentation as the first such comment wad.

Example:

\begin{verbatim}
(let ((x 10))
  (incf x 20)    ; The user wanted some distance here.
                 ; But there might be more to say.
  (+ x 30))
\end{verbatim}

If the first comment wad following the expression wad in a complete
indentation block is not preceded by the expression wad on the same
line, then the desired indentation for the comment wad is two spaces
past the end of the expression block.

Example:
\begin{verbatim}
(let ((x 10))
  (incf x 20)    
              ; For some reason, the user put this
              ; comment on a separate line.
  (+ x 30))
\end{verbatim}

Before we state the rules for degenerate indentation blocks, we need
to define what an inter-block indentation rule is.

\subsubsection{Inter-block indentation rules}

As mentioned previously, a compound wad has a bunch of rules
associated with it, and those rules determine the desired indentation
of child wads.  But a component wad also has another bunch of rules
that determine desired indentation of \emph{potential} wads to be
inserted between two existing child wads, or after the last child
wad.  Such a rule depends on the number and nature of preceding and
following expression types.

An inter-block rule is used in these situations:

\begin{enumerate}
\item When the cursor is positioned on a blank line between two
  complete indentation blocks inside a compound wad.
\item When the cursor is positioned after the last complete
  indentation block of a compound wad, is preceded by whitespace on
  the same line, and there is no degenerate indentation block in this
  compound wad.
\item For the wads of a degenerate indentation block.
\end{enumerate}

Examples:

\begin{verbatim}
;;; In this example, we think the rule for LET should
;;; indicate that a body form might be about to be typed
(let ((x 10))
  |
  (+ x 20))

;;; In this example, assuming the previous desired
;;; indentation of a declaration, perhaps another  
;;; declaration is coming up.
(let ((x 10))
   |
   (declare (type integer x))
  (+ x 20))

;;; Here, clearly another body form is coming:  
(let ((x 10))
   (declare (type integer x))
  (+ x 20)
  |)
\end{verbatim}

Some wads representing expressions such as strings, characters and
real numbers do not have children, so for this type of wad, only the
overall indentation is defined.

The component indentation of a compound wad depends only on the
overall indentation of that wad, and is therefore independent of the
ancestors of that wad.

If a wad is preceded by items other than whitespace on the same line,
then the desired indentation of the wad is the current indentation of
the wad.

For the purpose of this chapter, a buffer is considered to contain a
number of top-level wads, the last of which may be \emph{incomplete},
in that an attempt to read it reaches the end of the buffer before the
wad is complete.  When the cursor is at the end of the buffer, and the
last wad in the buffer is incomplete, for the purpose of indentation,
the cursor is considered to be at the end of the innermost incomplete
expression.

There are two basic cases for computing indentation of some wad:

\begin{enumerate}
\item If the wad is a top-level wad, then the
  indentation is always $0$, i.e. it should be positioned in the
  leftmost column.
\item If the wad is a nested wad, then the
  indentation is relative to some ancestor wad.
\end{enumerate}

For each top-level wad, its absolute indentation is first set
to $0$ and then the function \texttt{compute-child-indentations} is
called on that top-level wad.

The function \texttt{compute-child-indentations} takes two parameters,
a wad and a \emph{client} instance.

We must distinguish between the following cases:

\begin{enumerate}
\item $P$ is an atomic wad.
\item $P$ represents a special form or a function or macro call with
  its own indentation rule.
\item $P$ represents a function call without its own indentation rule.
\item $P$ represents a macro call without its own indentation rule.
\end{enumerate}

In case 1, we are done because its indentation has already been
computed and it has no children.  The remaining cases are treated
below.

The different cases are recognized because the function
\texttt{function-information} of the Cleavir system
\texttt{cst-to-ast} is called with the operator as an argument and the
return value of that function determines whether the operator is a
function, a macro, or a special operator.

\subsection{Special indentation rules}

Certain operators have their own indentation rules.  In particular,
most special operators do, but it is also possible to define special
indentation rules for functions and macros.

Such an indentation rule is defined as a method on the
generic-function named \texttt{compute-child-indentations-special}.
This function takes the wad $P$, a symbol (the name of the
operator), and a \emph{client} instance.  A method should specialize
on the name of the operator (in the form of an EQL specializer) and on
the client parameter (unless this is a default method).

The method computes and assigns indentation for the descendants that
have fixed relative indentation according to the syntax of the
operator, and then recursively calls
\texttt{compute-child-indentations} in order to recursively compute
indentation for the children of those descendants.

\subsection{Indenting a function call}

If $P$ represents a function call, then by default, its children are
indented according to the following rules:

\begin{itemize}
\item If $P$ has at least two children and the second child is
  positioned on the same line as the first child, then the remaining
  children (starting with the third one) are indented so that they
  align with the second child.
\item If either $P$ has only one child (which must then be the
  function to be called), or the second child is not positioned on the
  same line as the first child, then every child is indented by two
  positions relative to $P$.
\end{itemize}

However, some special cases exist.  For example, when the lambda list
of the function has keyword arguments, then keywords are aligned
vertically.%
\footnote{Provide a more detailed description.}

\subsection{Indenting a macro call}

The indentation of a macro call depends on the lambda list for the
macro.  Two major cases are identified:

\begin{itemize}
\item The lambda list does not have \texttt{\&body} in it.
\item The lambda list has \texttt{\&body} in it.
\end{itemize}

In the first case, the macro call is indented in the same way as a
function call.

In the second case, there are two sub-cases:

\begin{itemize}
\item If any of the arguments of the body is positioned on the same
  line as the operator, then every child that is not positioned on the
  same line as the operator is aligned under the first body argument.
\item Otherwise, every child that belongs to the body is indented by
  two positions relative to $P$.
\end{itemize}

If there is a child of $P$ preceding the body arguments and the first
such child is positioned on the same line as the operator, then every
child line preceding the body arguments is indented below the first
child.  Otherwise, if there is a child of $P$ preceding the body
arguments, then every line of such a child is indented by four spaces.

\subsection{Indenting lambda lists}
